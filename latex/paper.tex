\documentclass[conference]{IEEEtran}
\usepackage{graphicx}
\usepackage{times}

\begin{document}

\title{Six ways to understand monads\\
\small{Experiences from the functional programming course at TU Delft}}

\author{\IEEEauthorblockN{Georgios Gousios}
\IEEEauthorblockA{
Delft University of Technology\\
Delft, The Netherlands\\
Email: G.Gousios@tudelft.nl} 
\and
\IEEEauthorblockN{Erik Meijer}
\IEEEauthorblockA{Microsoft \\
Rendmond, WA, USA\\
Email: H.J.M.Meijer@tudelft.nl} 
}   
\maketitle

\begin{abstract}

  In the last few years, the popularity of functional programming as a way
  of solving computational problems has increased significantly. While most
  computer science curricula do include a course on functional programming,
  in many cases it is disconnected from practical applications, which is
  precisely where functional programming shines. To fill in this gap, we 
  designed a functional programming course that demanded from students to learn
  by experience with real world applications. In this work, we present our
  experiences with conducting this course.
\end{abstract}

\section{Introduction}

Due to a variety of reasons, including the advent of cloud computing, the rising
rate of information production and the necessity to reach the market fast,
currently, large corporations and start-ups alike are investigating
alternative programming and information storage models. As a result, during the
last few years, the practical software engineering field is witnessing a noticeable shift towards functional programming. Scripting languages,
notably Javascript and Ruby, pioneered the introduction of functional
concepts, such as closures and lambda functions, to mainstream programming. A
new wave of programming languages, developed to overcome the expressiveness and
complexity limitations exhibited in mainstream languages, have promoted
functional constructs, such as type safe pattern matching, higher order
functions and single assignment variables, to first class citizens (Scala). New,
purely functional, languages have emerged to fill in the remaining gaps (F\#,
Clojure), often introducing significant advancements in their field of
specialisation (such as Erlang in distributed fault-tolerant systems). Finally,
large scale information processing systems such as Map-Reduce and domain
specific languages such as LINQ have integrated functional concepts to ease the
expression of computations.

Broadly speaking, functional programming is a style of programming in which the
primary method of computation is the application of functions to arguments.
Among other features, functional languages offer a compact notation for writing
programs, powerful abstraction methods for structuring them, and a simple
mathematical basis that supports reasoning. Many of the advanced techniques in
modern functional languages, such as monads and catamorphisms, are closely based
on principles from category theory such as functors, initial algebras, monads
and Kleisli categories.

While functional programming has been taught for long in computer science
departments~\cite{Joost93}, curricula tend to emphasize functional programming theory rather
than practical applications. Special programming languages are used to teach
functional programming specific concepts, while little connection is made to how
those concepts can be transfered to non-functional languages.  The course at TU
Delft attempted to teach functional programming theory in the classroom and then
expect students to apply those concepts in the (non-purely functional) language
of their choice. The course had a very strong teaching by example focus:
students were expected to participate in both in-classroom exercises, homework
assignments and implement a real world system as a final project.

\section{The Course}

The aim of the course is to teach the principles of pure functional
programming, and the corresponding Category theoretical principles, using the
Haskell programming language. More specifically, the educational purposes
of the course were:

\begin{itemize}

  \item To introduce students to basic functional programming concepts, such
    as higher order functions, monads and type inference.

  \item To introduce students to solving data processing problems in a 
    functional way.

  \item To explain the application of functional concepts in non-purely
    functional environments.

\end{itemize}

The course consisted of a series of lectures 

\subsection{Student profile}

The course was elective at the MSc curriculum. Students from all departments in
the Electrical Engineering, Mathematics and Computer science faculty were
allowed to participate. Most students had received formal introduction to
imperative and object oriented programming in their bachelors curriculum, while
through participation to other courses they had limited exposure to functional
programming (i.e. Map/Reduce in the web information systems course). Two of the
students were majoring in computer engineering, which meant that their
programming experience was somewhat limited. Several students also had work
experience as programmers as part of their industrial placement or through
participation to start up ventures.

For organizational reasons, the participation to the course was limited to 15
students.

\subsection{Lectures}

\subsection{Student projects}

At the end of the lecture period, the students were given a selection of
projects to work on. The projects included:

\begin{itemize}

  \item Real time graph visualisation on steaming data. The
    particular example that students worked on was visualizing community
    structures for Github projects.

  \item Multi-source real time data processing. Students worked on a 
    programming language popularity index 

  \item Implementation of Haskell constructs in Javascript. Students implemented
    the Haskell prelude (basic functions for list manipulation). 

  \item Implementation of Map/Reduce algorithms in Haskell. Students installed
    and configured cloud Haskell and implemented simple Map/Reduce based
    algorithms in a distributed setting.

  \item Machine learning algorithms. Students implemented Na\"ive Bayes 
    classifiers and K-means clusters.

\end{itemize}

\section{Experiences}

\subsection{Teaching Monads}

\subsection{The teapot experiment}

One of the highlights of the taught period of the course was the teapot
experiment. While not a formal experiment in the scientific sense of the word,
we used it to draw the student's attention to the following facts:

\begin{itemize}

  \item Most modern programming languages can express functional programming
    constructs.

  \item Functional programming works best in data transformation scenarios

\end{itemize}

The exercise consisted of rendering the Utah Teapot~\cite{Torre06} using any
graphics primitive of the student's choice using only right triangles.  In its
core, the exercise required students to decompose arbitrary triangles, which
comprised the input Utah Teapot model, to a series right triangles.  The
students had to come up with the decomposition method (using some form of
analytic geometry), a decomposition restriction criterion to stop the
decomposition when triangles could not be rendered on screen and a method to
recursively apply the above mentioned transformations on the input data. As
always the students could decide the implementation language of their choice.

The students' response to the exercise was overwhelming. For four days,
the students worked hard to 

\subsection{Teaching Map/Reduce}

\subsection{Student Projects}

\section{Conclusion}

\section*{Acknowledgements}

We would like to thank\ldots

\bibliography{paper}
\bibliographystyle{ieeetr}
\end{document}

